%% abtex2-modelo-projeto-pesquisa.tex, v-1.9.6 laurocesar
%% Copyright 2012-2016 by abnTeX2 group at http://www.abntex.net.br/ 
%%
%% This work may be distributed and/or modified under the
%% conditions of the LaTeX Project Public License, either version 1.3
%% of this license or (at your option) any later version.
%% The latest version of this license is in
%%   http://www.latex-project.org/lppl.txt
%% and version 1.3 or later is part of all distributions of LaTeX
%% version 2005/12/01 or later.
%%
%% This work has the LPPL maintenance status `maintained'.
%% 
%% The Current Maintainer of this work is the abnTeX2 team, led
%% by Lauro César Araujo. Further information are available on 
%% http://www.abntex.net.br/
%%
%% This work consists of the files abntex2-modelo-projeto-pesquisa.tex
%% and abntex2-modelo-references.bib
%%

% ------------------------------------------------------------------------
% ------------------------------------------------------------------------
% abnTeX2: Modelo de Projeto de pesquisa em conformidade com 
% ABNT NBR 15287:2011 Informação e documentação - Projeto de pesquisa -
% Apresentação 
% ------------------------------------------------------------------------ 
% ------------------------------------------------------------------------

\documentclass[
	% -- opções da classe memoir --
	12pt,				% tamanho da fonte
	openright,		% capítulos começam em pág ímpar (insere página vazia caso preciso)
	twoside,			% para impressão em recto e verso. Oposto a oneside
	a4paper,			% tamanho do papel. 
	% -- opções da classe abntex2 --
	%chapter=TITLE,		% títulos de capítulos convertidos em letras maiúsculas
	%section=TITLE,		% títulos de seções convertidos em letras maiúsculas
	%subsection=TITLE,	% títulos de subseções convertidos em letras maiúsculas
	%subsubsection=TITLE,% títulos de subsubseções convertidos em letras maiúsculas
	% -- opções do pacote babel --
	english,				% idioma adicional para hifenização
	french,				% idioma adicional para hifenização
	spanish,			% idioma adicional para hifenização
	brazil,				% o último idioma é o principal do documento
	]{abntex2}

% ---
% PACOTES
% ---

% ---
% Pacotes fundamentais 
% ---
\usepackage{lmodern}				% Usa a fonte Latin Modern
\usepackage[T1]{fontenc}		% Selecao de codigos de fonte.
\usepackage[utf8]{inputenc}	% Codificacao do documento (conversão automática dos acentos)
\usepackage{indentfirst}			% Indenta o primeiro parágrafo de cada seção.
\usepackage{color}					% Controle das cores
\usepackage{graphicx}			% Inclusão de gráficos
\usepackage{microtype} 			% para melhorias de justificação
% ---

% ---
% Pacotes adicionais, usados apenas no âmbito do Modelo Canônico do abnteX2
% ---
\usepackage{lipsum}				% para geração de dummy text
% ---

% ---
% Pacotes de citações
% ---
\usepackage[brazilian,hyperpageref]{backref}	% Paginas com as citações na bibl
\usepackage[alf]{abntex2cite}							% Citações padrão ABNT

% --- 
% CONFIGURAÇÕES DE PACOTES
% --- 

% ---
% Configurações do pacote backref
% Usado sem a opção hyperpageref de backref
\renewcommand{\backrefpagesname}{Citado na(s) página(s):~}
% Texto padrão antes do número das páginas
\renewcommand{\backref}{}
% Define os textos da citação
\renewcommand*{\backrefalt}[4]{
	\ifcase #1 %
		Nenhuma citação no texto.%
	\or
		Citado na página #2.%
	\else
		Citado #1 vezes nas páginas #2.%
	\fi}%
% ---

% ---
% Informações de dados para CAPA e FOLHA DE ROSTO
% ---
\titulo{Modelo de Mineração de Dados \\
Análise das Perguntas do Site Stack Overflow}
\autor{Jefferson Carlos de Mendonça}
\local{São Paulo}
\data{Maio 2016, v-1.0.0}
\instituicao{%
  Escola Politécnica da Universidade de São Paulo -- USP
  \par
  Faculdade de Engenharia e Computação
  \par
  Programa de Pós-Graduação}
\tipotrabalho{Tese (Mestrado)}
% O preambulo deve conter o tipo do trabalho, o objetivo, 
% o nome da instituição e a área de concentração 
\preambulo{Modelo canônico de Projeto de pesquisa em conformidade
com as normas ABNT apresentado ao Programa de Pós-Graduação em Engenharia da Computação da Escola Politécnica da Universidade de São Paulo, requisito para a seleção de ingresso ao Curso de Mestrado }
% ---

% ---
% Configurações de aparência do PDF final

% alterando o aspecto da cor azul
\definecolor{blue}{RGB}{41,5,195}

% informações do PDF
\makeatletter
\hypersetup{
     	%pagebackref=true,
		pdftitle={\@title}, 
		pdfauthor={\@author},
    	pdfsubject={\imprimirpreambulo},
	    pdfcreator={LaTeX with abnTeX2},
		pdfkeywords={abnt}{latex}{abntex}{abntex2}{projeto de pesquisa}, 
		colorlinks=true,       		% false: boxed links; true: colored links
    	linkcolor=blue,          	% color of internal links
    	citecolor=blue,        		% color of links to bibliography
    	filecolor=magenta,      	% color of file links
		urlcolor=blue,
		bookmarksdepth=4
}
\makeatother
% --- 

% --- 
% Espaçamentos entre linhas e parágrafos 
% --- 

% O tamanho do parágrafo é dado por:
\setlength{\parindent}{1.3cm}

% Controle do espaçamento entre um parágrafo e outro:
\setlength{\parskip}{0.2cm}  % tente também \onelineskip

% ---
% compila o indice
% ---
\makeindex
% ---

% ----
% Início do documento
% ----
\begin{document}

% Seleciona o idioma do documento (conforme pacotes do babel)
%\selectlanguage{english}
\selectlanguage{brazil}

% Retira espaço extra obsoleto entre as frases.
\frenchspacing 

% ----------------------------------------------------------
% ELEMENTOS PRÉ-TEXTUAIS
% ----------------------------------------------------------
% \pretextual

% ---
% Capa
% ---
\imprimircapa
% ---

% ---
% Folha de rosto
% ---
\imprimirfolhaderosto
% ---

% ---
% NOTA DA ABNT NBR 15287:2011, p. 4:
%  ``Se exigido pela entidade, apresentar os dados curriculares do autor em
%     folha ou página distinta após a folha de rosto.''
% ---

% ---
% inserir lista de ilustrações
% ---
%%\pdfbookmark[0]{\listfigurename}{lof}
%%\listoffigures*
%%\cleardoublepage
% ---

% ---
% inserir lista de tabelas
% ---
\pdfbookmark[0]{\listtablename}{lot}
\listoftables*
\cleardoublepage
% ---

% ---
% inserir lista de abreviaturas e siglas
% ---
\begin{siglas}
  \item[USP] Universidade de São Paulo
  \item[API] Application Programming Interface
  \item[REST] Representational State Transfer
\end{siglas}
% ---

% ---
% inserir lista de símbolos
% ---
%%\begin{simbolos}
%%  \item[$ \Gamma $] Letra grega Gama
%%  \item[$ \Lambda $] Lambda
%%  \item[$ \zeta $] Letra grega minúscula zeta
%%  \item[$ \in $] Pertence
%% \end{simbolos}
% ---

% ---
% inserir o sumario
% ---
\pdfbookmark[0]{\contentsname}{toc}
\tableofcontents*
\cleardoublepage
% ---


% ----------------------------------------------------------
% ELEMENTOS TEXTUAIS
% ----------------------------------------------------------
\textual

% ----------------------------------------------------------
% Introdução
% ----------------------------------------------------------
\chapter*[Introdução]{Introdução}
\addcontentsline{toc}{chapter}{Introdução}

Não raro a evolução na área computacional se desenvolve em ritmo acelarado, novos algoritmos, técnicas para programação distribuída, inteligência artificial etc. Com as \emph{Linguagens de Programação} não é diferente, elas aprimoram suas API's constantemente, construindo e descontruindo métodos, muitas vezes quebrando paradigmas. O que se sabe hoje pode estar obsoleto amanhã, as exigências e tendênicas mudam com frequência, requerindo capacitação atualizada aos profissionais que atuam neste universo.  

O grande desafio das universidades que trabalham na formação de quadro profissionais é prover conhecimento não só para os especialistas com perfil para atuar em linhas de pesquisa, onde o conhecimento é mais profundo, mas também para aqueles que irão, após sua formação compor o mercado de trabalho, que muitas vezes é mais razo, porém mais dinâmico.

E para dar vazão a este dinamismo profissionais da área de computação diariamente recorrem a cursos online, livros, revistas e claro \textit{websites}. Dentre estas mídias merece destaque o fórum \href{http://stackoverflow.com/company/about}{\textit{Stack Overflow}}, maior comunidade online para programadores aprender, compartilhar conhecimento e progredir na carreira. Também é possível fazer um \textit{tour} pelo site\footnote{\url{http://stackoverflow.com/tour}}, onde são apresentados as regras e a mecânica desta poderosa ferramenta para troca de conhecimento.

O objeto deste projeto de pesquisa é propor um algoritmo de linguagem de máquina que consiga catalogar as perguntas desta comunidade, objetivando entender os principais questionamentos dos usuários e detectando padrões nas dúvidas mais frequentes. Estes insumos estarão disponíveis para que universidades e instituições de ensino possam aperfeiçoar seus cursos e treinamentos, estreitando a distância entre o que é lecionado e os requisitos impostos pelo dia-a-dia nas empresas.

% ----------------------------------------------------------
% Capitulo de textual  
% ----------------------------------------------------------
\chapter{Objetivos}

\chapterprecis {
Tópicos de interesse, assunto previsto, relevância para a área específica e aplicabilidade do estudo}\index{sinopse de capítulo}

\index{Objetivos} Os principais questionamentos dos usuários participantes da rede \textit{Stack Overflow}, objeto de pesquisa deste trabalho, será embasado pelos assuntos das áreas: \textsf{\textit{Educational Data Mining}}, \textsf{\textit{Learning Analytics}} e \textsf{\textit{Machine Learning}} e versará sobre os tópicos de interesse: \emph{Mineração de dados} para reconhecer padrões nas dúvidas dos usuários e \emph{algorítmos baseados em aprendizagem de máquina} para arranjar as perguntas sintáticamente equivalentes em grupos. 

O universo de dados coletados e categorizados, será de grande relevância  para instuições de ensino sobre o tema Programação de Computadores - foco das análises, que poderão aprimorar sua grade curricular e com isso aproximar a distância entre as expectativas do mercado de trabalho com o que é lecionado em salas de aula.


%%Consulte as demais normas da série ``Informação e documentação'' da ABNT
%%para outras informações. Uma lista com as principais normas dessa série, todas
%%observadas pelo \abnTeX, é apresentada em \citeonline{abntex2classe}.

% ----------------------------------------------------------
% Capitulo com exemplos de comandos inseridos de arquivo externo 
% ----------------------------------------------------------

%% abtex2-modelo-include-comandos.tex, v-1.9.6 laurocesar
%% Copyright 2012-2016 by abnTeX2 group at http://www.abntex.net.br/ 
%%
%% This work may be distributed and/or modified under the
%% conditions of the LaTeX Project Public License, either version 1.3
%% of this license or (at your option) any later version.
%% The latest version of this license is in
%%   http://www.latex-project.org/lppl.txt
%% and version 1.3 or later is part of all distributions of LaTeX
%% version 2005/12/01 or later.
%%
%% This work has the LPPL maintenance status `maintained'.
%% 
%% The Current Maintainer of this work is the abnTeX2 team, led
%% by Lauro César Araujo. Further information are available on 
%% http://www.abntex.net.br/
%%
%% This work consists of the files abntex2-modelo-include-comandos.tex
%% and abntex2-modelo-img-marca.pdf
%%

% ---
% Este capítulo, utilizado por diferentes exemplos do abnTeX2, ilustra o uso de
% comandos do abnTeX2 e de LaTeX.
% ---
 
\chapter{Metodologia}\label{cap_exemplos}

O ponto de partida será a reunião de material literário para a composição da estrutura e argumentação sobre o objeto de estudo proposto.

Na prática será desenvolvido um programa de computador para consumir os dados do \textit{Stack Overflow} utilizando sua API\footnote{\url{https://api.stackexchange.com/docs}}, imputando toda a informação extraída em uma fila de processamento.

Na segunda etapa será proposto um algoritmo de linguagem de máquina \cite{Hulth:2003:IAK:1119355.1119383} para extrair os assuntos principais \cite{Turney:2000:LAK:593957.593993} das dúvidas postadas no \textit{site} e armazenar as informações obtidas. Em adição também serão armazados a quantidade de vizualizações, pontuação ou número de respostas da referida pergunta.  

Com os dados processados e armazenados em um banco para grandes massa de dados\footnote{De acordo com a empresa \textit{Stack Overflow} o \textit{site} recebe cerca de 101 milhões de visitantes únicos mensalmente e conta com 3.7 milhões de perguntas respondidas}, faremos uso de técnicas de mineração de dados para detectar padrões nas dúvidas mais frequentes da comunidade e por fim será criado um \textit{ranking} \cite{mihalcea-tarau:2004:EMNLP} das mais relevantes para o ensino da Linguagem de Programação \emph{Java}. 

% ---
\chapter{Recursos}
Para que os algoritmos propostos, bem como os programas de computadores desenvolvidos para este projeto sejam executados em ambiente de alta performance, serão utilizados servidores na nuvem nas seguintes etapas: consumo dos \textit{endpoints} \emph{REST} providos pelo \textit{Stack Overflow} e entrada de dados na fila de processamento, \textit{queue}.
% ---

% ---
\chapter{Cronograma}
% ---

\index{tabelas}A \autoref{tab-nivinv} apresenta o cronograma estimado para a conclusão do projeto proposto.

\begin{table}[htb]
\ABNTEXfontereduzida
\caption[Cronograma]{Lista de atividades.}
\label{tab-nivinv}
\begin{tabular}{p{3.75cm}|p{9.0cm}|p{1.5cm}}
  %\hline
   \textbf{Atividade} & \textbf{Tópico}  & \textbf{Tempo}  \\
    \hline
    Linguagem de Máquina & Algoritmos para extração de textos & 32 \\
    \hline
    Mineração de Dados & Tabulação de dados & 32  \\
    \hline
    \textit{Big Data} & Técnicas de \textit{Map Reduce} & 24  \\
	\hline
    Infra & Criar infraestrutura na nuvem & 2  \\
    \hline
    \textit{Stack Overflow} & Estudo da API da comunidade & 4  \\
    \hline
    Programa & Extração dos dados via API & 4  \\    
    \hline
    Monografia & Revisar e complementar a monografia desenvolvida & 3  \\    
     \hline
    Apresentação & Material para apresentar a dissertação & 3  \\        
   % \hline
\end{tabular}
\legend{Observação: Tempo Estimado em semanas}
\end{table}







% ---
% Finaliza a parte no bookmark do PDF
% para que se inicie o bookmark na raiz
% e adiciona espaço de parte no Sumário
% ---
\phantompart

% ---
% Conclusão
% ---
\chapter*[Considerações finais]{Considerações finais}
\addcontentsline{toc}{chapter}{Considerações finais}

\lipsum[31-33]

% ----------------------------------------------------------
% ELEMENTOS PÓS-TEXTUAIS
% ----------------------------------------------------------
\postextual

% ----------------------------------------------------------
% Referências bibliográficas
% ----------------------------------------------------------
\bibliography{abntex2-modelo-references}

% ----------------------------------------------------------
% Glossário
% ----------------------------------------------------------
%
% Consulte o manual da classe abntex2 para orientações sobre o glossário.
%
%\glossary

%---------------------------------------------------------------------
% INDICE REMISSIVO
%---------------------------------------------------------------------

\phantompart

\printindex


\end{document}
