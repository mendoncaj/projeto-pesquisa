%% abtex2-modelo-include-comandos.tex, v-1.9.6 laurocesar
%% Copyright 2012-2016 by abnTeX2 group at http://www.abntex.net.br/ 
%%
%% This work may be distributed and/or modified under the
%% conditions of the LaTeX Project Public License, either version 1.3
%% of this license or (at your option) any later version.
%% The latest version of this license is in
%%   http://www.latex-project.org/lppl.txt
%% and version 1.3 or later is part of all distributions of LaTeX
%% version 2005/12/01 or later.
%%
%% This work has the LPPL maintenance status `maintained'.
%% 
%% The Current Maintainer of this work is the abnTeX2 team, led
%% by Lauro César Araujo. Further information are available on 
%% http://www.abntex.net.br/
%%
%% This work consists of the files abntex2-modelo-include-comandos.tex
%% and abntex2-modelo-img-marca.pdf
%%

% ---
% Este capítulo, utilizado por diferentes exemplos do abnTeX2, ilustra o uso de
% comandos do abnTeX2 e de LaTeX.
% ---
 
\chapter{Metodologia}

O ponto de partida será a reunião de material literário para a composição da estrutura e argumentação sobre o objeto de estudo proposto.

Na prática será desenvolvido um programa de computador para consumir os dados do \textit{Stack Overflow} utilizando sua API\footnote{\url{https://api.stackexchange.com/docs}}, imputando toda a informação extraída em uma fila de processamento.

Na segunda etapa será proposto um algoritmo de linguagem de máquina \cite{Hulth:2003:IAK:1119355.1119383} para extrair os assuntos principais \cite{Turney:2000:LAK:593957.593993} das dúvidas postadas no \textit{site} e armazenar as informações obtidas. Em adição também serão armazados a quantidade de vizualizações, pontuação e o número de respostas da referida pergunta.  

Com os dados processados e armazenados em um banco para grandes massa de dados\footnote{De acordo com a empresa \textit{Stack Overflow} o \textit{site} recebe cerca de 101 milhões de visitantes únicos mensalmente e conta com 3.7 milhões de perguntas respondidas}, faremos uso de técnicas de mineração de dados para detectar padrões nas dúvidas mais frequentes da comunidade e por fim será criado um \textit{ranking} \cite{mihalcea-tarau:2004:EMNLP} das mais relevantes para o ensino da Linguagem de Programação \emph{Java}. 

% ---
\chapter{Recursos}
Para que os algoritmos propostos, bem como os programas de computadores desenvolvidos para este projeto sejam executados em ambiente de alta performance, serão utilizados servidores na nuvem nas seguintes etapas: consumo dos \textit{endpoints} \emph{REST} providos pelo \textit{Stack Overflow} e entrada de dados na fila de processamento, \textit{queue}.
% ---

% ---
\chapter{Cronograma}
% ---	

\index{tabelas}A \autoref{tab-nivinv} apresenta o cronograma estimado para a conclusão do projeto proposto.

\begin{table}[htb]
\ABNTEXfontereduzida
\caption[Cronograma]{Lista de atividades.}
\label{tab-nivinv}
\begin{tabular}{p{3.75cm}|p{9.0cm}|p{1.5cm}}
  %\hline
   \textbf{Atividade} & \textbf{Tópico}  & \textbf{Tempo}  \\
    \hline
    Linguagem de Máquina & Algoritmos para extração de textos & 32 \\
    \hline
    Mineração de Dados & Tabulação de dados & 32  \\
    \hline
    \textit{Big Data} & Técnicas de \textit{Map Reduce} & 24  \\
	\hline
    Infra & Criar infraestrutura na nuvem & 2  \\
    \hline
    \textit{Stack Overflow} & Estudo da API da comunidade & 4  \\
    \hline
    Programa & Extração dos dados via API & 4  \\    
    \hline
    Monografia & Revisar e complementar a monografia desenvolvida & 3  \\    
     \hline
    Apresentação & Material para apresentar a dissertação & 3  \\        
   % \hline
\end{tabular}
\legend{Observação: Tempo Estimado em semanas}
\end{table}





